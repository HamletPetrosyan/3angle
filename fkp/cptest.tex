\documentclass[a4paper,15pt,titlepage]{article}

\usepackage{geometry}
\usepackage{indentfirst}
\usepackage{titlesec}
\usepackage[latin]{armtex}
\usepackage[utf8]{inputenc}
\usepackage{fancyhdr}
\usepackage{tabularx}

\def\hy{\armtm}

\geometry{
	a4paper,
	total={170mm, 253mm},
	left=20mm,
	top=22mm,
}

\titleformat{\section}
  {\normalfont\LARGE\bfseries}{}{0.3em}{}
\titleformat{\subsection}
  {\normalfont\Large\bfseries}{}{0.3em}{}

\pagestyle{fancy}
\fancyhf{}
\lhead{\hy{Հանրապետական փուլ} 2020-2021}
\rhead{\ttfamily{torder}}
\cfoot{\thepage}

\title{\hy{Բինար ծառի շրջանցումը}}
\author{}
\date {}

% Սկիզբ
\begin{document}

\section{\hy{Բինար ծառի շրջանցումը}}

\armsl {Ժամանակի սահմանափակում - }4\hy{ վ}

\armsl {Հիշողության սահմանափակում - }512\hy{ ՄԲ}
\newline

\hy{Տրված է $n$ գագաթ պարունակող բինար ծառ, որտեղ $1$ համարով գագաթը ծառի արմատն է։ Ծառի գագաթներից յուրաքանչյուրի վրա գրված է $1$-ից $n$ միջակայքի ինչ-որ թիվ, և ավելին` բոլոր գագաթներում գրված են տարբեր թվեր։}

\hy{Բինար ծառի շրջանցման հաջորդականությունը կսահմանենք ռեկուրսիվ ձևով․}
\begin{itemize}
    \item \hy{Վերցնենք ծառի արմատի ձախ զավակի ենթածառի շրջանցման հաջորդականությունը (եթե ձախ զավակը գոյություն չունի, կվերցնենք դատարկ հաջորդականություն)}
    \item \hy{Վերցնենք ծառի արմատում գրված թիվը}
    \item \hy{Վերցնենք ծառի արմատի աջ զավակի ենթածառի շրջանցման հաջորդականությունը (եթե աջ զավակը գոյություն չունի կվերցնենք դատարկ հաջորդականություն)}
\end{itemize}

\hy{Այսպիսով, բինար ծառի շրջանցման հաջորդականությունը կլինի վերը նշված $3$ հաջորդականությունների կցումը միմյանց։}

\hy{Ինչպես բոլորս գիտենք, բինար ծառում կամայական գագաթ ունի աջ և ձախ զավակ, բայց այս խնդրում ձեզ տրված է հնարավորություն ընտրելու, թե որ զավակն է ձախը, իսկ որը աջը։ Այսպիսով ձեր խնդիրն է, կամայական գագաթի համար այնպես որոշել աջ և ձախ զավակներին այնպես, որ բինար ծառի շրջանցման հաջորդականության ինվերսիաների քանակը լինի հնարավորինս քիչ։}

\hy{Հիշեցնենք, որ $S$ հաջորդականության ինվերսիաների քանակը հավասար է այն $(i, j)$ զույգերի քանակին, որտեղ $1 \leq i < j \leq size(S)$, իսկ $S_i > S_j$:}

\subsection{\hy{Մուտքային տվյալներ}}

\hy{Առաջին տողում տրված է մեկ բնական թիվ՝ $n$. բինար ծառի գագաթների քանակը։
Երկրորդ տողում տրված են իրարից մեկական բացակով բաժանված $n$ բնական թվեր՝ $a_1, a_2, ..., a_n$. գագաթներում գրված թվերը (բոլոր թվերը իրարից տարբեր են)։}

\hy{Հաջորդ $n - 1$ տողերից $i$-րդում տրված է երկու բնական թիվ՝ $v_i$ և $u_i$․ բինար ծառի կող։}

\subsection{\hy{Ելքային տվյալներ}}

\hy{Պետք է արտածել մեկ թիվ՝ բինար ծառի շրջանցման հաջորդականության ինվերսիաների քանակի հնարավոր մինիմալ արժեքը։}

\subsection{\hy{Օրինակներ}}

\noindent \begin{ttfamily} \begin{tabularx}{\textwidth}{|X|X|}
\hline
\armsl{Մուտքային տվյալներ}                                                             & \armsl{Ելքային տվյալներ}                                   \\ \hline
\begin{tabular}[t]{@{}l@{}}7\\ 1 6 7 4 5 2 3\\ 1 2\\ 5 2\\ 3 1\\ 7 3\\ 3 6\\ 2 4\end{tabular} & \begin{tabular}[t]{@{}l@{}}8\end{tabular} \\ \hline
\end{tabularx} \end{ttfamily}

\subsection{\hy{Սահմանափակումներ}}
\vspace{-3mm}
\noindent \begin{itemize}
\setlength\itemsep{1mm}
\item $1 \leq n \leq 10^5$
\item $1 \leq a_i \leq n$
\item $1 \leq v_i, u_i \leq n \ (v_i \neq u_i)$
\end{itemize}

\subsection{\hy{Ենթախնդիրներ}}

\def\arraystretch{1.5}
\noindent \begin{tabularx}{\textwidth}{llX}
0. & $(0 \ \hy{միավոր})$  & \hy{Օրինակները} \\
1. & $(7 \ \hy{միավոր})$  & \hy{$n \leq 1000$ և տրված բինար ծառը հանդիսանում է շղթա, որտեղ $i$-ն միացված է $i + 1$-ին, կամայական $1 \leq i \leq n$ համար}           \\
2. & $(13 \ \hy{միավոր})$ & \hy{Տրված բինար ծառը հանդիսանում է շղթա, որտեղ $i$-ն միացված է $i + 1$-ին, կամայական $1 \leq i < n$ համար}          \\
3. & $(10 \ \hy{միավոր})$ & $n \leq 15$          \\
4. & $(17 \ \hy{միավոր})$ & $n \leq 300$           \\
5. & $(13 \ \hy{միավոր})$ & $n \leq 2000$           \\
6. & $(40 \ \hy{միավոր})$ & \hy{Հավելյալ սահմանափակումներ չկան}          
\end{tabularx}

\end{document}
